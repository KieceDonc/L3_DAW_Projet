\documentclass[a4paper,10pt]{article}
%\documentclass[a4paper,10pt]{scrartcl}
\usepackage{graphicx}
\usepackage[utf8]{inputenc}
\usepackage{listings}
\usepackage{hyperref}
\usepackage{indentfirst}
\usepackage{float}

\title{
Rapport de développement d'applications Web\\
Projet 2022 : Création d'un site de formation
}
\author{
Bertoux Hugo, Hamidou Nazim, Verstracte Valentin\\
Petit Evan, Perion Maxence, Pinon Alexandre
}
\date{}

\renewcommand*\contentsname{Sommaire}

\begin{document}

\maketitle 
\tableofcontents

\newpage
\section{Introduction}
L'objectif de ce projet a été de réaliser un site de formation destiné à des apprenants/étudiants. Nous avons basé la construction de notre site autour d'un jeu vidéo nommé ``League Of Legend" et c'est pour cette raison que nous avons décidé de l'appeler ``E-lolning". 
Dans ce rapport, nous allons aborder tous les aspects de la conception de notre site, c'est-à-dire que nous allons aussi bien expliquer la modélisation, le modèle MVC utilisé et tout ce qui peut-être fait sur notre site. Il est important de rappeler qu'un site de formation en ligne ou site d'e-learning dans notre cas pour un jeu vidéo, permet aux utilisateurs de celui-ci de développer leurs compétences en participant à différents cours et discuter à propos de ce même cours dans un forum entre apprenants. Dans notre cas, nous donnons également la possibilité de créer un cours à qui le souhaite afin de partager son savoir et ses connaissances. 

\section{La gestion du projet}
Qui a fait quoi ? Comment on a repartie les taches ? Reunion ? Trello qui a permis d'organiser l'ensemble

\section{La modélisation}
Diagramme use case
Diagramme de classe

\section{Notre installation}
Serveur Lamp chez Valentin

\section{La charte graphique}
Les couleurs et le texte choisir avec le site : https://coolors.co/

\section{L'architecture MVC}
\subsection{Modèle}
\subsection{Vue}
\subsection{Contrôleur}

\section{Les cours}
\subsection{Quizz}
\subsection{Forum}
Le forum permet la discussion entre apprenants. Il est global et les utilisateurs peuvent créer un nouveau sujet ou répondre dans un déjà créé s'ils sont connectés. Ils peuvent également consulter la liste des sujets et les messages dans ceux-ci. Les utilisateurs connectés ont également la possibilité de modifier ou supprimer les sujets et les messages dont ils sont les auteurs.

\section{Les apprenants}

\section{Les administrateurs}


\section{Les options}
\subsection{Multilingue}
FR et EN pour tous les utilisateurs 

\subsection{Thèmes}
Thèmes claire et sombre

\subsection{Décompte des visiteurs}
Compteur du nombre de visiteur via fichier et non BDD.

\section{Conclusion}


\end{document}
