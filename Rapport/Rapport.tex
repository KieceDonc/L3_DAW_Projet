\documentclass[a4paper,10pt]{article}
%\documentclass[a4paper,10pt]{scrartcl}
\usepackage{graphicx}
\usepackage[utf8]{inputenc}
\usepackage{listings}
\usepackage{hyperref}
\usepackage{indentfirst}
\usepackage{float}

\title{
Rapport de développement d'applications Web\\
Projet 2022 : Création de site de formation
}
\author{
Bertoux Hugo, Hamidou Nazim, Verstracte Valentin\\
Petit Evan, Perion Maxence, Pinon Alexandre
}
\date{}

\renewcommand*\contentsname{Sommaire}

\begin{document}

\maketitle 
\tableofcontents

\newpage
\section{Introduction}
L'objectif de ce projet a été de réaliser un site de formation destiné à des apprenants/étudiants. Nous avons basé la construction de notre site autour d'un jeu vidéo nommée "League Of Legend", c'est pour cette raison que nous avons décidé d'appeler notre site "E-lolning". 
Durant ce rapport nous allons aborder tous les aspects de la conception de notre site, c'est-à-dire que nous allons aussi bien expliquer la modélisation, ainsi que le modèle MVC utilisé et tous ce qui peut-être fait sur notre site. Il est important de rappeler qu'un site de formation en ligne ou site d'e-learning dans notre cas pour un jeu vidéo, permet aux utilisateurs de celui-ci de développer leurs compétences en participants à différents cours et discuter à propos de ce même cours, mais également en ayant la possibilité de créer un cours afin de partager sont savoir et ses connaissances. 

\section{La gestion du projet}
Qui a fait quoi ? Comment on a repartie les taches ? Reunion ? Trello qui a permis d'organiser l'ensemble

\section{La modélisation}
Diagramme use case
Diagramme de classe

\section{La charte graphique}
Les couleurs et le texte choisir avec le site : ???

\section{L'architecture MVC}

\section{Les cours}
\subsection{Quizz}
\subsection{Forum}


\section{Les apprenants}

\section{Les administrateurs}


\section{Les options}
\subsection{Multilingue}
FR et EN pour tous les utilisateurs 

\subsection{Thèmes}
Thèmes claire et sombre

\section{Conclusion}


\end{document}
